\documentclass[parskip=half]{scrartcl}

\usepackage{tikz}
\usetikzlibrary{positioning, shapes, patterns}
\tikzset{
    state/.style = {draw, align=center, minimum width = 2.1cm, rectangle split, rectangle split parts = 3, thick, minimum height = 5cm},
    tip/.style = {thick, ->, >=stealth}
}
\usepackage{wrapfig}

\usepackage{fontspec}
\usepackage{unicode-math}
\usepackage{microtype}
\setmainfont{Libertinus Serif}
\setsansfont{Libertinus Sans}
\setmathfont{Libertinus Math}

% maths
\usepackage{IEEEtrantools}

% tables
\usepackage{booktabs}

\usepackage{listings}
\lstset{
    breaklines = true,
    %breakatwhitespace = true,
    language=[5.3]Lua,
    backgroundcolor = \color{blue!10!white},
    basicstyle = \small\ttfamily,
    keywordstyle = \color{blue},
    commentstyle = \color{red},
    stringstyle = \color{green!70!black},
    showstringspaces = false,
    tabsize = 4,
    gobble = 4,
    emph = {
        layout,
        parameters,
        pointarray,
        path,
        geometry.rectangle, 
        geometry.multiple, 
        geometry.path, 
        pcell.setup, 
        pcell.process_args, 
        pcell.check_args,
        pcell.add_parameters,
        pcell.get_parameters,
        pcell.push_overwrites,
        pcell.pop_overwrites,
        object.translate,
        object.merge_into_shallow,
        get_anchor,
        move_anchor,
        translate,
        merge_into_shallow,
        flipx,
        flipy,
        add_child,
        add_child_reference,
        add_child_link,
        generics,
        generics.metal,
        generics.via,
        generics.contact,
        generics.other,
        generics.mapped,
        util.xmirror,
        util.make_insert_xy
    },
    emphstyle = \color{blue!60!green}\bfseries,
    belowskip=-0.2\baselineskip
}
% -------------------------------------------------------------------------------------------------------------
% taken from https://tex.stackexchange.com/questions/48903/how-to-extend-the-lstinputlisting-command
\errorcontextlines=\maxdimen

\newlength{\rawgobble}
\newlength{\gobble}
% Remove a single space
\settowidth{\rawgobble}{\small\ttfamily\ }
\setlength{\rawgobble}{-\rawgobble}

\makeatletter
\lst@Key{widthgobble}{0}{%
    % Reindent a bit by multiplying with 0.9, then multiply by tabsize and number of indentation levels
    %\setlength{\gobble}{0.9\rawgobble}%
    \setlength{\gobble}{#1\rawgobble}%
    %\setlength{\gobble}{\sepstartwo\gobble}%
    \def\lst@xleftmargin{\gobble}%
    \def\lst@framexleftmargin{\gobble}%
}
\makeatother
% -------------------------------------------------------------------------------------------------------------
\newcommand{\shellinline}[1]{\lstinline!#1!}
%\lstnewenvironment{shellcode}{\lstset{language=bash, keywordstyle = \relax, commentstyle = \relax, stringstyle = \relax, identifierstyle = \relax, breakautoindent = false}}{}
\lstnewenvironment{shellcode}{\lstset{keywordstyle = \relax, commentstyle = \relax, stringstyle = \relax, identifierstyle = \relax, breakautoindent = false}}{}
\newcommand{\luainline}[1]{\lstinline!#1!}
\newcommand{\lualisting}[2][]{\lstinputlisting[#1]{#2}}
%\newmintedfile[lualisting]{lua}{autogobble, fontsize = \small}
% API documentation commands
\newcommand{\param}[1]{\luainline{#1}}
\newenvironment{apifunc}[1]{\par\hspace*{-2em}\luainline{#1}\leftskip=2em\par}{\par}

\usepackage{siunitx}
\usepackage{csquotes}

\frenchspacing



\title{OpenPCells}
\subtitle{PCell Design Guide and API}
\author{Patrick Kurth}

\begin{document}
\maketitle
\begin{abstract}
    \noindent This is the official documentation of the OpenPCells project.  It is split in several different files for clarity. This document provides an overview
    of the creation of PCells in the OpenPCells environment as well as a detailed API documentation.  If you are looking for a general overview of the project and
    how to use it, start with the user guide, which also contains a tutorial for getting started quickly. If you want to know more about the technical details and
    implementation notes, look into the technical documentation.
\end{abstract}

\tableofcontents

\section{PCell Creation -- Introductory Examples}
We will start this documentation by a series of examples to show the main features and API functions. The to-be-created cells will get increasingly complex to
demonstrate various features.

Every cell is defined by a function where all shapes making up the shape are described. This function gets called by the cell generation system, which passes the
main object and a table with all defined parameters. The name for this function is \luainline{layout()}. Additional functions such as \luainline{parameters()}
are also understood.

\subsection{Simple Rectangle}
The first example is a simple rectangle of variable width and height. As mentioned, all the code for the rectangle resides in a function \luainline{layout()}.
The parameters of the cell are defined in a function \luainline{parameters()}, which is optional in theory, but since we're designing pcells, there is not much point
of leaving it out. In \luainline{layout()}, we receive the main object and the defined parameters. Here we can modify the object based on the parameters.

The simple rectangle looks like this:
\luafilelisting{code/simple_rectangle.lua}
Let's walk through this line-by-line (sort of). First, we declare the function for the parameter definition:
\luafilelisting[firstline = 2, lastline = 2]{code/simple_rectangle.lua}
In the function, we add the parameters, here we use the width and the height of the rectangle:
\luafilelisting[widthgobble=4, firstline = 3, lastline = 6]{code/simple_rectangle.lua}
We can add as many parameters as we like (\luainline{pcell.add_parameters()} accepts any number of arguments). For every argument, the first entry in the table is
the name of the parameter, the second entry is the default value. This is the simplest form, we can supply more information for finer control. We will see some
examples of this later on.

The default value for both parameters is 100, which is a \emph{size}, meaning it has a unit. Physical/geometrical parameters like width or height are specified
in nanometers.\footnote{Well, this is not entirely sure. Only integers are allowed and the base unit is assumed to be nanometer. This is also
currently reflected for example in the GDSII export, where the scaling is done approriately. However, it is planned that this will change in the
future, making the base unit in opc arbitrary.}

This is all for the \luainline{parameters()} function, so let's move on to \luainline{layout()}. This functions takes two arguments: the main object that will be
placed in the layout and the table with parameters for the cell (which already includes any parsed arguments given before the cell creation).

We can name them in any way that pleases us, the common name used in all standard cells distributed by this project is \luainline{_P} (as hommage to the global
environment \luainline{_G} in lua).
Of course it is possible to \enquote{unpack} the parameters, storing them in individual variables, but for cells with many parameters this rather is a bloat.
\luafilelisting[widthgobble = 0, firstline = 10, lastline = 10]{code/simple_rectangle.lua}

Now that we have all the layout parameters, we can already create the rectangle:
\luafilelisting[widthgobble = 4, firstline = 12, lastline = 12]{code/simple_rectangle.lua}
There is a lot going on here: We use the \luainline{geometry.rectangle} function to create a rectangle with with and height (third and fourth argument).
Since we are creating shapes of IC geometry, we have to specify a layer.
But we also want to create technology-independent pcells, so there is a generics system for layers.
Right now we are just using the \luainline{generics.metal} function, which takes a single number as argument.
\luainline{generics.metal(1)} specifies the first metal (counted from silicon), you can also say something like \luainline{generics.metal(-2)}, where \luainline{-1} is the index of the highest metal.
Lastly we pass the main object as first argument to the function, which places the rectangle within this object.

%\luainline{merge_into_shallow} is a \emph{method} of objects and needs a little explaining of the object system.
%As mentioned earlier, every geometry function creates what is called an object.
%An object is a collection of shapes, where each shape is made up of a layer-purpose-pair and points (which can currently be interpreted as rectangles or polygons).
%The cell generation systems expects all created shapes to be merged into the main object passed as argument (\luainline{obj} in this case).
%The function postfix \emph{shallow} is a hint towards cell hierarchies, where cells contain children, which can again have children and so on.
%Shallow in this case means that we only regard shapes that are in the referenced cell, not in the children.

This cell can now be created by calling the main program with an appropriate export and technology. Note that there's another manual about that, so we won't get
into any details here. The simplest call would be something like:
\begin{shellcode}
    opc --technology opc --export gds --cell simple_rectangle
\end{shellcode}

\subsection{Array of Rectangles}
Now that we know how to create a rectangle, we want to create an entire array, that is a rectangular area made up of several individual rectangles. This could be
used for example as filling. We will setup the cell exactly as before, we only have to add two new parameters: the repetition and the pitch (we will start with
quadratic arrays with equal pitch in both directions):
\luafilelisting[firstline = 1, lastline = 8]{code/rectangle_array.lua}
The default arguments are \num{200} for the pitch and \num{10} for the number of repetitions, which creates a ten-by-ten array of rectangles with a spacing of
\num{100} and a width and height of \num{100}. Again, remember that we work in nanometers here.

For the repetition we could use a loop to create the individual rectangles, but we also have to move them to their correct position.
The \luainline{geometry.rectangle} function takes a shift in x and y direction as fifth and sixth argument.
We have to calculate the correct offset, which results in:
\luafilelisting[widthgobble = 4, firstline = 12, lastline = 21]{code/rectangle_array.lua}
This loop is cumbersome as it has to be set up and furthermore the calculation of the offset for the rectangles is done by hand.
The function \luainline{geometry.rectangle} already can do what we need, as it can take the repetition and pitch in x and y as additional arguments.
This places repeated copies of a template rectangle (with width and height) with their center in the origin.
With it, we can replace the whole loop construct with:
\luafilelisting[widthgobble = 4, firstline = 24, lastline = 28]{code/rectangle_array.lua}
With this, the whole cell looks like this:
\luafilelisting{code/rectangle_array.lua}
Now you already now how to create simple rectangles and rectangle arrays with generic layers.
As integrated circuits are mostly made up of rectangles, one can already built a surprising amount of pcells.
However, we have to discuss how we can create layers other than metals, vias and shapes with more complex outlines than rectangles.
We will talk about that in the remaining cell tutorials.

\subsection{Metal-Oxide-Metal Capacitors}
Many technologies don't have so-called metal-insulator-metal capacitors (mimcaps), so the standard way to implement capacitors is be using interdigitated metals.
Let's do that. As before, we set up the pcell. Useful parameters are the number of fingers, the width and height of the fingers and the spacing in between.
Furthermore, we shift one collection of fingers (one plate) up and the other down to separate them and connect them together.
Lastly, we also specify the width of the connecting rails and the used metals:
\luafilelisting[firstline = 1, lastline = 12]{code/momcap.lua}
The parameter definition also shows how you can use better names for displaying: Simply write them in parantheses. When querying the defined parameters of a
cell, the display names are used, but within the cell the regular names are significant. This enables easier syntax: \luainline{_P.fingers} as opposed to
\luainline{_P["Number of Fingers"]}.

In \luainline{layout()} we  loop over all metals to draw the fingers.
We don't have to create every finger separately, with \luainline{geometry.rectangle} and the multiple capabilities this becomes very simple.
Since the upper and lower fingers are one-off and \luainline{geometry.rectangle} centeres all objects, we only have to move them a little bit up/down.
This is done with the corresponding arguments to \luainline{geometry.rectangle} (note that \luainline{//} denotes floor division in lua):
\luafilelisting[widthgobble = 4, firstline = 18, lastline = 31]{code/momcap.lua}
We create two arrays of fingers, one for the \enquote{upper plate}, one for the \enquote{lower plate}. All fingers have the same width, height and pitch. For the
upper plate, we use one more finger, the placement in \luainline{geometry.rectangle} automatically arranges them centered, so that this \enquote{just works}.  The
\param{ypitch} for \luainline{geometry.rectangle} is \num{0}, which is ok since we only have a \param{yrep} of \num{1}.

The rails connecting the fingers are created in a similar manner:
\luafilelisting[widthgobble = 4, firstline = 33, lastline = 40]{code/momcap.lua}

What remains is the drawing of the vias between the metals.
For this we introduce a new \luainline{geometry} function: \luainline{geomtry.via}.
It takes two arguments -- one for the start- and one for the end-metal for the via stack. We don't have to specify the individual vias between each layer in the stack, this
is resolved later by the technology translation. The vias are placed in the rails:
\luafilelisting[widthgobble = 4, firstline = 42, lastline = 49]{code/momcap.lua}
With this the pcell is finished, the entire listing is in \texttt{code/momcap.lua}.
A cell similar to this is bundled in this release of openPCells (\texttt{cells/passive/capacitor/mom.lua}).
A few optimizations and additional parameters are added, but the here shown implementation is the basic structure of this capacitor.

\subsection{Octagonal Inductor}
RF designs often require on-chip inductors, which usually are built in an octagonal shape due to angle restrictions in most technologies (no true circles or
better approximations available). We will show how to built a differential (symmetric) octagonal inductor with a variable number of turns (integers). We will
skip some basic techniques that we already discussed a few times such as setting up the cell body, cell parameters and main object. Look into
\texttt{cells/passive/inductor/octagonal.lua} for the defined parameters.

An inductor is basically a wire routed in a special manner, therefor we will describe the inductor as a \luainline{path}. This is a series of points that gets
converted into a polygon with a \luainline{width}. To create a path, we have to pass the points, which we will store in a \luainline{table}. Here is how this looks
for the octagonal inductor:
\luafilelisting[widthgobble = 8, firstline = 31, lastline = 40]{../../../cells/passive/inductor/octagonal.lua}
\luainline{util.make_insert_xy} is a helper function, that returns a function that appends/prepends points to an array. It's purpose is to simplify code, one
might as well just use \luainline{table.insert}.

This is just an excerpt from the cell, the entire code generating the path points is a bit complex and involves some mathematical thoughts. Since this tutorial is
about how to build the code for cells, the actual points will not be discussed.

After the points are assembled, we can create the path. The cell only draws half of the inductor, so we draw the path twice, one time with mirrored points (notice
\luainline{util.xmirror(pathpts)} in the second line):
\luafilelisting[widthgobble = 8, firstline = 90, lastline = 95]{../../../cells/passive/inductor/octagonal.lua}
The \luainline{geometry.path} function takes four arguments: the layer, the points of the path, the width and whether to use a miter- or a bevel-join. Bevel-join is
default, so \luainline{true} is specified for a miter-join.

\subsection{References and Inheritance}
In order to be able to build larger layouts, cells must be reused in hierarchies (for instance, a current mirror is made up of several transistors). It would be
a good decision to build everything from scratch. Therefore, openPCells offers some basic support for such things. We will look at logic gates to illustrate the
different options. All gates are built from transistors, so we will assume for now that there is a cell to place one. Furthermore, digital designs mostly
(always?) use a few geometry parameters for all cells, such as the gate length. It makes sense to store this in one place so we can redefine it for all cells in
a hierarchy, if we want to change that. In the supplied logic cell family (\texttt{cells/stdcells}), this is handled by the \texttt{stdcells/base} cell. It is a abstract
cell, that is, it does not define a layout function, so it can't be called. But it does store the relevant parameters, which get referenced by the top cells (and
temporarily changed). Other cells then access the parameters, for instance \texttt{stdcells/not\_gate.lua}:
\luafilelisting[firstline = 1, lastline = 11]{../../../cells/stdcells/not_gate.lua}
This cell has only has a small number of parameters but uses the supplied parameters of \texttt{stdcells/base.lua} for the layout function. The parameters are used
implicitly by creating other cells that reference these parameters, and explicitly by accessing them with \luainline{pcell.get_parameters}. In order to do this,
a cell has to reference that cell with \luainline{pcell.reference_cell}. The referenced parameters reflect the current values, that is if a top cell instantiates the
inverter, it can overwrite the values of referenced parameters, affecting all sub cells. This is achieved by calling \luainline{pcell.push_overwrites}, which can
be seen in \texttt{stdcells/1\_inv\_gate.lua} (basic cell for and/or):
\luafilelisting[widthgobble = 4, firstline = 13, lastline = 13]{../../../cells/stdcells/1_inv_gate.lua}
Cells that need to prevent parameters being changed from the toplevel have to include guards against that, which is exactly what can be seen in the previous
listing. The overwrites are stack-based, so the last \luainline{push_overwrites} counts. In \texttt{stdcells/1\_inv\_gate.lua}, rightdummies can not be changed from
outside anymore.
\subsection{Translation, Object Placement and the Alignment Box}
We already saw the \luainline{translate} object method to move objects. For more complex layouts, a better method exists. Cells can have anchors, that is, points
at certain places in the layout which can be referenced. You can ask for the location of an anchor from an object (\luainline{get_anchor}) and you can move a
cell so that the specified anchor lies at a certain point (\luainline{move_anchor}). Let's see an example:
\luafilelisting[widthgobble = 8, firstline = 62, lastline = 64]{../../../cells/stdcells/harness.lua}
The used anchors have to be present in the specific cell, and it is up to the designer of that cell to provide the needed anchors.

Of course do anchors move if an object is moved, but they also move if the cell is flipped/mirror/rotated etc. This ensures that the anchors are always at the
right place in the layout. However, in certain cases, the opposite behaviour is useful: If you want to, say, place two digital standard cells next to each other,
you can use appropriate anchors for left and right (for instance at the leftmost and rightmost source/drain) and align them at these points. This works fine
until a cell is mirrored at the y-axis. Then the anchor left becomes right and the other way around. This means that cells generating such layouts need to handle
flipped/rotated/mirrored cells differently, which is a nuisance. There is a special set of anchors for exactly this problem: the alignment box. This box is also
defined by the cell designer and in effect it provides the special anchors left, right, top, bottom and sensible combinations of those (e.g. topleft, NOT
leftright). The alignment box undergoes translation, but not rotation/flipping/mirroring. This allows the following (from \texttt{stdcells/dff.lua}):
\luafilelisting[widthgobble = 4, firstline = 75, lastline = 78]{../../../cells/stdcells/dff.lua}
Notice the calls in the last two lines, where the cell gets flipped in x direction (mirrored at the y axis) but still the left anchor is used for alignment (the
cell is placed right of the isogate).

The above example shows another important fact about object transformation: Translation is always applied last (after rotation/flipping/mirroring). This is
contrary to many other graphical programs, where for instance rotation is applied with the origin is center. In opc, the order of the transformation statements
does not matter. This is intentional, as this is what is usually needed in layouts and simplifies the pcell code. The other behaviour can always be achieved by
appropriate translation.

\subsection{Cell Hierarchies}
Layouts of integrated circuits usually make great use of repetition/reuse of cells. For instance, a shift register uses the same flip flop over and over again.
Creating \emph{flat} layouts (that is, layouts without any hierarchies) for these cells can be quite ressource-intense, as more shapes have to be calculated by
opc and the resulting layout is very likely to be larger in file size than a hierarchical one. Therefore, opc supports hierachical layouts. For every sub-cell
(called a child), a reference is needed that defines the shapes of this child. A child link then connects to that reference. We can see a simple example of this
in \texttt{stdcells/1\_inv\_gate.lua}:
\luafilelisting[widthgobble = 4, firstline = 14, lastline = 16]{../../../cells/stdcells/1_inv_gate.lua}
\luainline{add_child} is an object method and a short cut for creating a reference and the link. For exports, that support children (like GDS), the layout will
now be hierarchical with one toplevel cell with two children.

The above example showed how instead of \luainline{merge_into_shallow}, \luainline{add_child} is used to add another cell to a layout. For simple layouts, the
advantage is mostly negligible, but it pays out for larger layouts. In \texttt{analog/current\_starved\_ringoscillator.lua}, multiple children links are placed:
\luafilelisting[widthgobble = 4, firstline = 186, lastline = 193]{../../../cells/analog/current_starved_ringoscillator.lua}
First, the reference has to be established with an object and a name (\luainline{add_child_reference}). This function returns the actual name that is used to
refer to this cell, as the name has to be unique. With this name, child links can be added with \luainline{add_child_link}, which returns an proxy object. This
proxy object behaves like a regular object and knows the same methods, but it has no shapes or children of its own.

\section{Available PCells}
In the following subsections, all available cells will be documented. The current status is rather a poor one, but work is ongoing.
\subsection{Transistor}
The transistor might be the most important cell and currently it's also definitely the most complex one. Therefor, this documentation starts with a description of
the goal. Figure \ref{fig:transistor} shows an example with all geometrical parameters, a summary of all parameters can be found in table \ref{tab:transistor}. The
cell draws a number of gates on top of an active area (with some implant/well/etc. markers).
\begin{figure}[htb]
    \centering
    \definecolor{activegreen}{RGB}{0,204,102}
    \begin{tikzpicture}
        [
            %marker/.style = {draw = yellow, pattern = dots, pattern color = yellow},
            %active/.style = {draw = activegreen, pattern = grid, pattern color = activegreen},
            %gate/.style = {draw = red, pattern = crosshatch, pattern color = red},
            %metal/.style = {draw = blue, pattern = crosshatch dots, pattern color = blue},
            marker/.style = {draw = none, fill = yellow, opacity = 0.5},
            active/.style = {draw = none, fill = activegreen},
            gate/.style = {draw = none, fill = red},
            metal/.style = {draw = none, fill = blue, opacity = 0.5},
            annotation/.style = {<->, >=stealth, very thick}
        ]
        \def\fingers{4}
        \def\flength{0.5}
        \def\fwidth{4}
        \def\fspace{2}
        \def\gtopext{1}
        \def\gbotext{1}
        \def\gatestrwidth{0.75}
        \def\sdwidth{0.8}
        \def\actext{1.0}
        % active marker
        \draw[marker] ({-0.5 * \fingers * (\flength + \fspace) - \actext}, {-0.5 * \fwidth - \actext}) rectangle
                      ({ 0.5 * \fingers * (\flength + \fspace) + \actext}, { 0.5 * \fwidth + \actext});
        % active
        \draw[active] ({-0.5 * \fingers * (\flength + \fspace)}, {-0.5 * \fwidth}) rectangle
                      ({ 0.5 * \fingers * (\flength + \fspace)}, { 0.5 * \fwidth});
        % active
        \draw[active] ({-0.5 * \fingers * (\flength + \fspace)}, {-0.5 * \fwidth}) rectangle ({0.5 * \fingers * (\flength + \fspace)}, {0.5 * \fwidth});
        % gates
        \foreach \x in {1, ..., \fingers}
        {
            \draw[gate] ({\fspace * (\x - 0.5 * (\fingers - 1) - 1) - 0.5 * \flength}, { -0.5 * \fwidth - \gbotext}) rectangle
                        ({\fspace * (\x - 0.5 * (\fingers - 1) - 1) + 0.5 * \flength}, {  0.5 * \fwidth + \gtopext});
        }
        % metal
        \draw[metal] ({-0.5 * \fingers * (\flength + \fspace)}, {-0.5 * \gatestrwidth + 0.5 * \fwidth + \gtopext}) rectangle
                     ({ 0.5 * \fingers * (\flength + \fspace)}, { 0.5 * \gatestrwidth + 0.5 * \fwidth + \gtopext});
        \draw[metal] ({-0.5 * \fingers * (\flength + \fspace)}, {-0.5 * \gatestrwidth - 0.5 * \fwidth - \gbotext}) rectangle
                     ({ 0.5 * \fingers * (\flength + \fspace)}, { 0.5 * \gatestrwidth - 0.5 * \fwidth - \gbotext});
        \foreach \x in {0, ..., \fingers}
        {
            \draw[metal] ({\fspace * (\x - 0.5 * \fingers) - 0.5 * \sdwidth}, { -0.5 * \fwidth}) rectangle
                         ({\fspace * (\x - 0.5 * \fingers) + 0.5 * \sdwidth}, {  0.5 * \fwidth});
        }
        % annotations
        \draw[annotation] ({-0.5 * \fingers * (\flength + \fspace)}, {-0.5 * \fwidth}) -- node[left] {fwidth} ({-0.5 * \fingers * (\flength + \fspace)}, {0.5 * \fwidth});
        \draw[annotation] ({\fspace * (1 - 0.5 * (\fingers - 1) - 1) - 0.5 * \flength}, {-0.25 * \fwidth}) -- node[below] {flength} ++(\flength, 0);
        \draw[annotation] ({\fspace * (1 - 0.5 * (\fingers - 1) - 1) + 0.5 * \flength}, { 0.25 * \fwidth}) -- node[below] {fspace}  ++({\fspace - \flength}, 0);
        \draw[annotation] ({\fspace * (\fingers - 0.5 * (\fingers - 1) - 1) + 0.5 * \flength}, { -0.5 * \fwidth - \gbotext}) -- node[right] {gbotext} ++(0,  \gbotext);
        \draw[annotation] ({\fspace * (\fingers - 0.5 * (\fingers - 1) - 1) + 0.5 * \flength}, {  0.5 * \fwidth + \gtopext}) -- node[right] {gtopext} ++(0, -\gbotext);
        \draw[annotation] ({\fspace * (3 - 0.5 * \fingers) - 0.5 * \sdwidth}, { 0.25 * \fwidth}) -- node[below] {sdwidth}  ++(\sdwidth, 0);
        \draw[annotation] ({ 0.5 * \fingers * (\flength + \fspace)}, 0) -- node[above] {actext} ++(\actext, 0);
    \end{tikzpicture}
    \caption{Overview of the transistor}
    \label{fig:transistor}
\end{figure}
Furthermore, it draws some metals and vias (not shown in figure \ref{fig:transistor}) in the source/drain regions and for gate contacts.

\begin{table}[htb]
    \centering
    \begin{tabular}{llc}
        \toprule
        Parameter & Meaning & Default \\
        \midrule
        channeltype     & Type of Transistor & "nmos" \\
        oxidetype       & Oxide Thickness Index & 1 \\
        vthtype         & Threshold Voltage Index & 1 \\
        fingers         & Number of Fingers& 4 \\
        fwidth          & Finger Width & 1.0 \\
        gatelength      & Finger Length & 0.15 \\
        fspace          & Space between Fingers & 0.27 \\
        actext          & Left/Right Extension of Active Area & 0.03 \\
        sdwidth         & Width of Source/Drain Metals & 0.2 \\
        sdconnwidth     & Width of Source/Drain Connection Rails Metal & 0.2 \\
        sdconnspace     & Space of Source/Drain Connection Rails Metal & 0.2 \\
        gtopext         & Gate Top Extension & 0.2 \\
        gbotext         & Gate Bottom Extension & 0.2 \\
        typext          & Implant/Well Extension around Active & 0.1 \\
        cliptop         & Clip Top Marking Layers (Implant, Well, etc.) & false \\
        clipbot         & Clip Bottom Marking Layers (Implant, Well, etc.) & false \\
        drawtopgate     & Draw Top Gate Strap & false \\
        drawbotgate     & Draw Bottom Gate Strap & false \\
        topgatestrwidth && 0.12 \\
        topgatestrext   && 1 \\
        botgatestrwidth && 0.12 \\
        botgatestrext   && 1 \\
        topgcut         & Draw Top Poly Cut & false \\
        botgcut         & Draw Bottom Poly Cut & false \\
        connectsource   & Connect all Sources together & false \\
        connectdrain    & Connect all Drains together & false\\
        \bottomrule
    \end{tabular}
    \caption{Summary of Transistor Parameters}
    \label{tab:transistor}
\end{table}

\section{API Documentation}
\subsection{geometry Module}
\begin{apifunc}{geometry.rectangle(layer, width, height)}
    Create a rectangular shape with a width of \param{width} and a height of \param{height} in the layer-purpose-pair \param{layer} (usually a generic). The function
    returns an object.
\end{apifunc}
\begin{apifunc}{geometry.multiple(obj, xrep, yrep, xpitch, ypitch)}
    Creates a rectangular array (mosaic) of an object with \param{xrep} repetitions in x and \param{yrep} repetitions in y. \param{xpitch} and \param{ypitch} are the
    center-to-center space in x and y direction. The entire array gets centered. The function returns the merged objects.
\end{apifunc}
\subsection{Object Module}
\subsection{Shape Module}
\subsection{Pointarray Module}
\subsection{Point Module}
\end{document}

% vim: ft=tex
