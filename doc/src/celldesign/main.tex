\documentclass[parskip=half]{scrartcl}

\usepackage{tikz}
\usetikzlibrary{positioning, shapes, patterns}
\tikzset{
    state/.style = {draw, align=center, minimum width = 2.1cm, rectangle split, rectangle split parts = 3, thick, minimum height = 5cm},
    tip/.style = {thick, ->, >=stealth}
}
\usepackage{wrapfig}

\usepackage{fontspec}
\usepackage{unicode-math}
\usepackage{microtype}
\setmainfont{Libertinus Serif}
\setsansfont{Libertinus Sans}
\setmathfont{Libertinus Math}

% maths
\usepackage{IEEEtrantools}

% tables
\usepackage{booktabs}

\usepackage{listings}
\lstset{
    breaklines = true,
    %breakatwhitespace = true,
    language=[5.3]Lua,
    backgroundcolor = \color{blue!10!white},
    basicstyle = \small\ttfamily,
    keywordstyle = \color{blue},
    commentstyle = \color{red},
    stringstyle = \color{green!70!black},
    showstringspaces = false,
    tabsize = 4,
    gobble = 4,
    emph = {
        layout,
        parameters,
        pointarray,
        path,
        geometry.rectangle, 
        geometry.multiple, 
        geometry.path, 
        pcell.setup, 
        pcell.process_args, 
        pcell.check_args,
        pcell.add_parameters,
        pcell.get_parameters,
        pcell.push_overwrites,
        pcell.pop_overwrites,
        object.translate,
        object.merge_into_shallow,
        get_anchor,
        move_anchor,
        translate,
        merge_into_shallow,
        flipx,
        flipy,
        add_child,
        add_child_reference,
        add_child_link,
        generics,
        generics.metal,
        generics.via,
        generics.contact,
        generics.other,
        generics.mapped,
        util.xmirror,
        util.make_insert_xy
    },
    emphstyle = \color{blue!60!green}\bfseries,
    belowskip=-0.2\baselineskip
}
% -------------------------------------------------------------------------------------------------------------
% taken from https://tex.stackexchange.com/questions/48903/how-to-extend-the-lstinputlisting-command
\errorcontextlines=\maxdimen

\newlength{\rawgobble}
\newlength{\gobble}
% Remove a single space
\settowidth{\rawgobble}{\small\ttfamily\ }
\setlength{\rawgobble}{-\rawgobble}

\makeatletter
\lst@Key{widthgobble}{0}{%
    % Reindent a bit by multiplying with 0.9, then multiply by tabsize and number of indentation levels
    %\setlength{\gobble}{0.9\rawgobble}%
    \setlength{\gobble}{#1\rawgobble}%
    %\setlength{\gobble}{\sepstartwo\gobble}%
    \def\lst@xleftmargin{\gobble}%
    \def\lst@framexleftmargin{\gobble}%
}
\makeatother
% -------------------------------------------------------------------------------------------------------------
\newcommand{\shellinline}[1]{\lstinline!#1!}
%\lstnewenvironment{shellcode}{\lstset{language=bash, keywordstyle = \relax, commentstyle = \relax, stringstyle = \relax, identifierstyle = \relax, breakautoindent = false}}{}
\lstnewenvironment{shellcode}{\lstset{keywordstyle = \relax, commentstyle = \relax, stringstyle = \relax, identifierstyle = \relax, breakautoindent = false}}{}
\newcommand{\luainline}[1]{\lstinline!#1!}
\newcommand{\lualisting}[2][]{\lstinputlisting[#1]{#2}}
%\newmintedfile[lualisting]{lua}{autogobble, fontsize = \small}
% API documentation commands
\newcommand{\param}[1]{\luainline{#1}}
\newenvironment{apifunc}[1]{\par\hspace*{-2em}\luainline{#1}\leftskip=2em\par}{\par}

\usepackage{siunitx}
\usepackage{csquotes}

\frenchspacing



\title{OpenPCells}
\subtitle{PCell Design Guide and API}
\author{Patrick Kurth}

\begin{document}
\maketitle
\begin{abstract}
    \noindent This is the official documentation of the OpenPCells project.  It is split in several different files for clarity. This document provides an overview
    of the creation of PCells in the OpenPCells environment as well as a detailed API documentation.  If you are looking for a general overview of the project and
    how to use it, start with the user guide, which also contains a tutorial for getting started quickly. If you want to know more about the technical details and
    implementation notes, look into the technical documentation.
\end{abstract}

\tableofcontents

\section{PCell Creation -- Introductory Examples}
We will start this documentation by a series of examples to show the main features and API functions. The to-be-created cells will get increasingly complex to
demonstrate various features.

Every cell is defined by a function where all shapes making up the shape are described. This function gets called by the cell generation system, which passes the
main object and a table with all defined parameters. The name for this function is \luainline{layout()}. Additional functions such as \luainline{parameters()}
are also understood.

\subsection{Simple Rectangle}
The first example is a simple rectangle of variable width and height. As mentioned, all the code for the rectangle resides in a function \luainline{layout()}.
The parameters of the cell are defined in a function \luainline{parameters()}, which is optional in theory, but since we're designing pcells, there is not much point
of leaving it out. In \luainline{layout()}, we receive the main object and the defined parameters. Here we can modify the object based on the parameters.

The simple rectangle looks like this:
\lualisting{code/simple_rectangle.lua}
Let's walk through this line-by-line (sort of). First, we declare the function for the parameter definition:
\lualisting[firstline = 2, lastline = 2]{code/simple_rectangle.lua}
In the function, we add the parameters, here we use the width and the height of the rectangle:
\lualisting[widthgobble=4, firstline = 3, lastline = 6]{code/simple_rectangle.lua}
We can add as many parameters as we like (\luainline{pcell.add_parameters()} accepts any number of arguments). For every argument, the first entry in the table is
the name of the parameter, the second entry is the default value. This is the simplest form, we can supply more information for finer control. We will see some
examples of this later on.

The default value for both parameters is 100, which is a \emph{size}, meaning it has a unit. Physical/geometrical parameters like width or height are specified
in nanometers.\footnote{Of course you can do what you want in a cell, but the modules that process the cells after creation work in nanometers. It is best
practice to do the same.}

This is all for the \luainline{parameters()} function, so let's move on to \luainline{layout()}. This functions takes two arguments: the main object that will be
placed in the layout and the table with parameters for the cell (which already includes any parsed arguments given before the cell creation). 

We can name them in any way that pleases us, the common name used in all standard cells distributed by this project is \luainline{_P} (as hommage to the global
environment \luainline{_G} in lua).
Of course it is possible to \enquote{unpack} the parameters, storing them in individual variables, but for cells with many parameters this rather is a bloat.
\lualisting[widthgobble = 0, firstline = 10, lastline = 10]{code/simple_rectangle.lua}

Now that we have all the layout parameters, we can already create the rectangle:
\lualisting[widthgobble = 4, firstline = 12, lastline = 13]{code/simple_rectangle.lua}
There is a lot going on here: We use the \luainline{geometry.rectangle} function to create a rectangle with with and height (second and third argument). Since we are
creating shapes of IC geometry, we have to specify a layer. But we also want to create technology-independent pcells, so there is a generics system for layers. Right
now we are just using the \luainline{generics.metal} function, which takes a single number as argument. \luainline{generics.metal(1)} specifies the first metal
(counted from silicon), you can also say something like \luainline{generics.metal(-2)}, where \luainline{-1} is the index of the highest metal. Lastly we save the return value
of \luainline{geometry.rectangle} in a local variable \luainline{rect}, which is a hint to the type: All geometry functions return \luainline{objects}, which has some
consequences for the use of these functions. We will get into that later.

That is all we have to do for the geometry of the cell, so we merge that into the main cell:
\lualisting[widthgobble = 4, firstline = 15, lastline = 15]{code/simple_rectangle.lua}
\luainline{merge_into} is a \emph{method} of objects and needs a little explaining of the object system. As mentioned
earlier, every geometry function creates what is called an object. An object is a collection of shapes, where each shape is made up of a layer-purpose-pair and
points (which can currently be interpreted as rectangles or polygons). The cell generation systems expects all created shapes to be merged into the main object
passed as argument (\luainline{obj} in this case).

This cell can now be created by calling the main program with an appropriate export and technology. Note that there's another manual about that, so we won't get
into any details here. The simplest call would be something like:
\begin{shellcode}
    opc --technology generic --export gds --cell simple_rectangle
\end{shellcode}

\subsection{Array of Rectangles}
Now that we know how to create a rectangle, we want to create an entire array, that is a rectangular area made up of several individual rectangles. This could be
used for example as filling. We will setup the cell exactly as before, we only have to add two new parameters: the repetition and the pitch (we will start with
quadratic arrays with equal pitch in both directions):
\lualisting[firstline = 1, lastline = 8]{code/rectangle_array.lua}
The default arguments are \num{200} for the pitch and \num{10} for the number of repetitions, which creates a ten-by-ten array of rectangles with a spacing of
\num{100} and a width and height of \num{100}. Again, remember that we work in nanometers here.

For the repetition we could use a loop to create the objects:
\lualisting[widthgobble = 4, firstline = 12, lastline = 19]{code/rectangle_array.lua}
In order for this to work, we also have to move the rectangles to the correct position, something that we didn't learn yet. This comes later, as this also involves
some math we don't want to talk about right now. Just keep in mind that the above loop is wrong and cumbersome. In any ways, there is a function that does
exactly what we want to achieve:
\luainline{geometry.multiple}. It takes an object as first argument and then the repetition in x and y and the pitch in x and y and returns an array of repeated
objects with the center in the origin. With it, we can replace the whole loop construct with:
\lualisting[widthgobble = 4, firstline = 22, lastline = 25]{code/rectangle_array.lua}
\luainline{geometry.multiple} also already merges all objects so we don't have to take care of that. Therefor, we receive a single object which we simply can
merge directly into the main cell. The whole cell looks like this:
\lualisting{code/rectangle_array.lua}
Now you already now how to create rectangles, with generic layers, \luainline{geometry.multiple} and object merging. With this, one can already built a
surprising amount of pcells. However, we have to discuss how we can create layers other than metals, vias and shapes with more complex outlines than rectangles. We
will talk about that in the remaining cell tutorials.

\subsection{Metal-Oxide-Metal Capacitors}
Many technologies don't have so-called metal-insulator-metal capacitors (mimcaps), so the standard way to implement capacitors is be using interdigitated metals.
Let's do that. As before, we set up the pcell. Useful parameters are the number of fingers, the width and height of the fingers and the spacing in between.
Furthermore, we shift one collection of fingers (one plate) up and the other down to separate them and connect them together. Lastly, we also specify the used
metals and the width of the connecting rails:
\lualisting[firstline = 1, lastline = 12]{../../../cells/passive/capacitor/mom.lua}
The parameter definition also shows how you can use better names for displaying: Simply write them in parantheses. When querying the defined parameters of a
cell, the display names are used, but within the cell the regular names are significant. This enables easier syntax: \luainline{_P.fingers} as opposed to
\luainline{_P["Number of Fingers"]}.

In \luainline{layout()} we  loop over all metals to draw the fingers. We don't have to create every finger separately, with \luainline{geometry.multiple} this
becomes very simple. Since the upper and lower fingers are one-off and \luainline{geometry.multiple} centeres all objects, we only have to move them a little bit
up/down. This is done with \luainline{object.translate} (a method of an object), taking x- and y-offset as arguments: 
\lualisting[widthgobble = 4, firstline = 15, lastline = 25]{../../../cells/passive/capacitor/mom.lua} 
We create two arrays of fingers, one for the \enquote{upper plate}, one for the \enquote{lower plate}. All fingers have the same width, height and pitch. For the
upper plate, we use one more finger, the placement in \luainline{geometry.multiple} automatically arranges them centered, so that this \enquote{just works}.  The
\param{ypitch} for \luainline{geometry.multiple} is \num{0}, which is ok since we only have a \param{yrep} of \num{1}. 

The rails connecting the fingers are created in a similar manner:
\lualisting[widthgobble = 4, firstline = 27, lastline = 33]{../../../cells/passive/capacitor/mom.lua}
The \luainline{end} delimits the \luainline{for}-loop.

What remains is the drawing of the vias between the metals. For this we introduce a new \luainline{generics} function: \luainline{generics.via}. It takes two
arguments for the start- and end-metal for the via stack. We don't have to specify the individual vias between each layer in the stack, this is resolved later by the
technology translation. The vias are placed in the rails:
\lualisting[widthgobble = 4, firstline = 34, lastline = 39]{../../../cells/passive/capacitor/mom.lua}
With this the pcell is finished, the entire listing is in \texttt{cells/passive/capacitor/mom.lua}.

\subsection{Octagonal Inductor}
RF designs often require on-chip inductors, which usually are built in an octagonal shape due to angle restrictions in most technologies (no true circles or
better approximations available). We will show how to built a differential (symmetric) octagonal inductor with a variable number of turns (integers). We will
skip some basic techniques that we already discussed a few times such as setting up the cell body, cell parameters and main object. Look into
\texttt{cells/passive/inductor/octagonal.lua} for the defined parameters.

An inductor is basically a wire routed in a special manner, therefor we will describe the inductor as a \luainline{path}. This is a series of points that gets
converted into a polygon with a \luainline{width}. To create a path, we have to pass the points, which we will store in a \luainline{table}. Here is how this looks
for the octagonal inductor: 
\lualisting[widthgobble = 8, firstline = 31, lastline = 40]{../../../cells/passive/inductor/octagonal.lua} 
\luainline{util.make_insert_xy} is a helper function, that returns a function that appends/prepends points to an array. It's purpose is to simplify code, one
might as well just use \luainline{table.insert}.

This is just an excerpt from the cell, the entire code generating the path points is a bit complex and involves some mathematical thoughts. Since this tutorial is
about how to build the code for cells, the actual points will not be discussed.

After the points are assembled, we can create the path. The cell only draws half of the inductor, so we draw the path twice, one time with mirrored points (notice
\luainline{util.xmirror(pathpts)} in the second line):
\lualisting[widthgobble = 8, firstline = 89, lastline = 94]{../../../cells/passive/inductor/octagonal.lua}
The \luainline{geometry.path} function takes four arguments: the layer, the points of the path, the width and whether to use a miter- or a bevel-join. Bevel-join is
default, so \luainline{true} is specified for a miter-join.

\subsection{References and Inheritance}
In order to be able to build larger layouts, cells must be reused in hierarchies (for instance, a current mirror is made up of several transistors). It would be
a good decision to build everything from scratch. Therefore, openPCells offers some basic support for such things. We will look at logic gates to illustrate the
different options. All gates are built from transistors, so we will assume for now that there is a cell to place one. Furthermore, digital designs mostly
(always?) use a few geometry parameters for all cells, such as the gate length. It makes sense to store this in one place so we can redefine it for all cells in
a hierarchy, if we want to change that. In the supplied logic cell family (\texttt{cells/logic}), this is handled by the \texttt{base} cell. It is a abstract
cell, that is, it does not define a layout function, so it can't be called. But it does store the relevant parameters, which get referenced by the top cells (and
temporarily changed). Other cells then access the parameters, for instance \texttt{not\_gate.lua}:
\lualisting[firstline = 1, lastline = 7]{../../../cells/logic/not_gate.lua}
This cell has only one parameter \texttt{fingers}, but uses the supplied parameters of \texttt{base.lua} for the layout function. These parameters reflect the
current values, that is if a top cell instantiates the inverter, it can overwrite the values of referenced parameters, affecting all sub cells. This is achieved
by calling \luainline{pcell.push_overwrites}:
\lualisting[widthgobble = 4, firstline = 12, lastline = 18]{../../../cells/logic/not_gate.lua}

\section{Available PCells}
In the following subsections, all available cells will be documented. The current status is rather a poor one, but work is ongoing.
\subsection{Transistor}
The transistor might be the most important cell and currently it's also definitely the most complex one. Therefor, this documentation starts with a description of
the goal. Figure \ref{fig:transistor} shows an example with all geometrical parameters, a summary of all parameters can be found in table \ref{tab:transistor}. The
cell draws a number of gates on top of an active area (with some implant/well/etc. markers). 
\begin{figure}[htb]
    \centering
    \definecolor{activegreen}{RGB}{0,204,102}
    \begin{tikzpicture}
        [
            %marker/.style = {draw = yellow, pattern = dots, pattern color = yellow},
            %active/.style = {draw = activegreen, pattern = grid, pattern color = activegreen},
            %gate/.style = {draw = red, pattern = crosshatch, pattern color = red},
            %metal/.style = {draw = blue, pattern = crosshatch dots, pattern color = blue},
            marker/.style = {draw = none, fill = yellow, opacity = 0.5},
            active/.style = {draw = none, fill = activegreen},
            gate/.style = {draw = none, fill = red},
            metal/.style = {draw = none, fill = blue, opacity = 0.5},
            annotation/.style = {<->, >=stealth, very thick}
        ]
        \def\fingers{4}
        \def\flength{0.5}
        \def\fwidth{4}
        \def\fspace{2}
        \def\gtopext{1}
        \def\gbotext{1}
        \def\gatestrwidth{0.75}
        \def\sdwidth{0.8}
        \def\actext{1.0}
        % active marker
        \draw[marker] ({-0.5 * \fingers * (\flength + \fspace) - \actext}, {-0.5 * \fwidth - \actext}) rectangle 
                      ({ 0.5 * \fingers * (\flength + \fspace) + \actext}, { 0.5 * \fwidth + \actext});
        % active
        \draw[active] ({-0.5 * \fingers * (\flength + \fspace)}, {-0.5 * \fwidth}) rectangle 
                      ({ 0.5 * \fingers * (\flength + \fspace)}, { 0.5 * \fwidth});
        % active
        \draw[active] ({-0.5 * \fingers * (\flength + \fspace)}, {-0.5 * \fwidth}) rectangle ({0.5 * \fingers * (\flength + \fspace)}, {0.5 * \fwidth});
        % gates
        \foreach \x in {1, ..., \fingers}
        {
            \draw[gate] ({\fspace * (\x - 0.5 * (\fingers - 1) - 1) - 0.5 * \flength}, { -0.5 * \fwidth - \gbotext}) rectangle 
                        ({\fspace * (\x - 0.5 * (\fingers - 1) - 1) + 0.5 * \flength}, {  0.5 * \fwidth + \gtopext});
        }
        % metal
        \draw[metal] ({-0.5 * \fingers * (\flength + \fspace)}, {-0.5 * \gatestrwidth + 0.5 * \fwidth + \gtopext}) rectangle 
                     ({ 0.5 * \fingers * (\flength + \fspace)}, { 0.5 * \gatestrwidth + 0.5 * \fwidth + \gtopext});
        \draw[metal] ({-0.5 * \fingers * (\flength + \fspace)}, {-0.5 * \gatestrwidth - 0.5 * \fwidth - \gbotext}) rectangle 
                     ({ 0.5 * \fingers * (\flength + \fspace)}, { 0.5 * \gatestrwidth - 0.5 * \fwidth - \gbotext});
        \foreach \x in {0, ..., \fingers}
        {
            \draw[metal] ({\fspace * (\x - 0.5 * \fingers) - 0.5 * \sdwidth}, { -0.5 * \fwidth}) rectangle 
                         ({\fspace * (\x - 0.5 * \fingers) + 0.5 * \sdwidth}, {  0.5 * \fwidth});
        }
        % annotations
        \draw[annotation] ({-0.5 * \fingers * (\flength + \fspace)}, {-0.5 * \fwidth}) -- node[left] {fwidth} ({-0.5 * \fingers * (\flength + \fspace)}, {0.5 * \fwidth});
        \draw[annotation] ({\fspace * (1 - 0.5 * (\fingers - 1) - 1) - 0.5 * \flength}, {-0.25 * \fwidth}) -- node[below] {flength} ++(\flength, 0);
        \draw[annotation] ({\fspace * (1 - 0.5 * (\fingers - 1) - 1) + 0.5 * \flength}, { 0.25 * \fwidth}) -- node[below] {fspace}  ++({\fspace - \flength}, 0);
        \draw[annotation] ({\fspace * (\fingers - 0.5 * (\fingers - 1) - 1) + 0.5 * \flength}, { -0.5 * \fwidth - \gbotext}) -- node[right] {gbotext} ++(0,  \gbotext);
        \draw[annotation] ({\fspace * (\fingers - 0.5 * (\fingers - 1) - 1) + 0.5 * \flength}, {  0.5 * \fwidth + \gtopext}) -- node[right] {gtopext} ++(0, -\gbotext);
        \draw[annotation] ({\fspace * (3 - 0.5 * \fingers) - 0.5 * \sdwidth}, { 0.25 * \fwidth}) -- node[below] {sdwidth}  ++(\sdwidth, 0);
        \draw[annotation] ({ 0.5 * \fingers * (\flength + \fspace)}, 0) -- node[above] {actext} ++(\actext, 0);
    \end{tikzpicture}
    \caption{Overview of the transistor}
    \label{fig:transistor}
\end{figure}
Furthermore, it draws some metals and vias (not shown in figure \ref{fig:transistor}) in the source/drain regions and for gate contacts.

\begin{table}[htb]
    \centering
    \begin{tabular}{llc}
        \toprule
        Parameter & Meaning & Default \\
        \midrule
        channeltype     & Type of Transistor & "nmos" \\
        oxidetype       & Oxide Thickness Index & 1 \\
        vthtype         & Threshold Voltage Index & 1 \\
        fingers         & Number of Fingers& 4 \\
        fwidth          & Finger Width & 1.0 \\
        gatelength      & Finger Length & 0.15 \\
        fspace          & Space between Fingers & 0.27 \\
        actext          & Left/Right Extension of Active Area & 0.03 \\
        sdwidth         & Width of Source/Drain Metals & 0.2 \\
        sdconnwidth     & Width of Source/Drain Connection Rails Metal & 0.2 \\
        sdconnspace     & Space of Source/Drain Connection Rails Metal & 0.2 \\
        gtopext         & Gate Top Extension & 0.2 \\
        gbotext         & Gate Bottom Extension & 0.2 \\
        typext          & Implant/Well Extension around Active & 0.1 \\
        cliptop         & Clip Top Marking Layers (Implant, Well, etc.) & false \\
        clipbot         & Clip Bottom Marking Layers (Implant, Well, etc.) & false \\
        drawtopgate     & Draw Top Gate Strap & false \\
        drawbotgate     & Draw Bottom Gate Strap & false \\
        topgatestrwidth && 0.12 \\
        topgatestrext   && 1 \\
        botgatestrwidth && 0.12 \\
        botgatestrext   && 1 \\
        topgcut         & Draw Top Poly Cut & false \\
        botgcut         & Draw Bottom Poly Cut & false \\
        connectsource   & Connect all Sources together & false \\
        connectdrain    & Connect all Drains together & false\\
        \bottomrule
    \end{tabular}
    \caption{Summary of Transistor Parameters}
    \label{tab:transistor}
\end{table}

\section{API Documentation}
\subsection{geometry Module}
\begin{apifunc}{geometry.rectangle(layer, width, height)} 
    Create a rectangular shape with a width of \param{width} and a height of \param{height} in the layer-purpose-pair \param{layer} (usually a generic). The function
    returns an object.
\end{apifunc}
\begin{apifunc}{geometry.multiple(obj, xrep, yrep, xpitch, ypitch)} 
    Creates a rectangular array (mosaic) of an object with \param{xrep} repetitions in x and \param{yrep} repetitions in y. \param{xpitch} and \param{ypitch} are the
    center-to-center space in x and y direction. The entire array gets centered. The function returns the merged objects.
\end{apifunc}
\subsection{Object Module}
\subsection{Shape Module}
\subsection{Pointarray Module}
\subsection{Point Module}
\end{document}

% vim: ft=tex
