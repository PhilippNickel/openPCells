\documentclass[parskip=half]{scrartcl}

\usepackage{tikz}
\usetikzlibrary{positioning, shapes, patterns}
\tikzset{
    state/.style = {draw, align=center, minimum width = 2.1cm, rectangle split, rectangle split parts = 3, thick, minimum height = 5cm},
    tip/.style = {thick, ->, >=stealth}
}
\usepackage{wrapfig}

\usepackage{fontspec}
\usepackage{unicode-math}
\usepackage{microtype}
\setmainfont{Libertinus Serif}
\setsansfont{Libertinus Sans}
\setmathfont{Libertinus Math}

% maths
\usepackage{IEEEtrantools}

% tables
\usepackage{booktabs}

\usepackage{enumitem}

\usepackage{listings}
\lstset{
    breaklines = true,
    %breakatwhitespace = true,
    language=[5.3]Lua,
    backgroundcolor = \color{blue!10!white},
    basicstyle = \small\ttfamily,
    keywordstyle = \color{blue},
    commentstyle = \color{red},
    stringstyle = \color{green!70!black},
    showstringspaces = false,
    tabsize = 4,
    gobble = 4,
    emph = {
        layout,
        parameters,
        % rectangle module
        geometry, 
        geometry.rectangle, 
        rectangle, 
        geometry.rectanglebltr, 
        rectanglebltr, 
        geometry.rectanglepoints, 
        rectanglepoints, 
        geometry.path, 
        path, 
        geometry.via, 
        via, 
        geometry.viabltr, 
        viabltr, 
        geometry.contact, 
        contact, 
        geometry.contactbltr, 
        contactbltr, 
        % pcell module
        pcell.reference_cell, 
        pcell.setup, 
        pcell.process_args, 
        pcell.check_args,
        pcell.add_parameters,
        pcell.get_parameters,
        pcell.push_overwrites,
        pcell.pop_overwrites,
        object.translate,
        object.merge_into_shallow,
        get_anchor,
        move_anchor,
        translate,
        merge_into_shallow,
        flipx,
        flipy,
        add_child,
        add_child_reference,
        add_child_link,
        generics,
        generics.metal,
        generics.via,
        generics.contact,
        generics.other,
        generics.mapped,
        util.xmirror,
        util.make_insert_xy
    },
    emphstyle = \color{blue!60!green}\bfseries,
    belowskip=-0.2\baselineskip
}
% -------------------------------------------------------------------------------------------------------------
% taken from https://tex.stackexchange.com/questions/48903/how-to-extend-the-lstinputlisting-command
\errorcontextlines=\maxdimen

\newlength{\rawgobble}
\newlength{\gobble}
% Remove a single space
\settowidth{\rawgobble}{\small\ttfamily\ }
\setlength{\rawgobble}{-\rawgobble}

\makeatletter
\lst@Key{widthgobble}{0}{%
    % Reindent a bit by multiplying with 0.9, then multiply by tabsize and number of indentation levels
    %\setlength{\gobble}{0.9\rawgobble}%
    \setlength{\gobble}{#1\rawgobble}%
    %\setlength{\gobble}{\sepstartwo\gobble}%
    \def\lst@xleftmargin{\gobble}%
    \def\lst@framexleftmargin{\gobble}%
}
\makeatother
% -------------------------------------------------------------------------------------------------------------
\newcommand{\shellinline}[1]{\lstinline!#1!}
\lstnewenvironment{shellcode}{\lstset{keywordstyle = \relax, commentstyle = \relax, stringstyle = \relax, identifierstyle = \relax, breakautoindent = false}}{}
\newcommand{\luainline}[1]{\lstinline!#1!}
\lstnewenvironment{lualisting}{}{}
\newcommand{\luafilelisting}[2][]{\lstinputlisting[#1]{#2}}

% API documentation commands
\newenvironment{APIfunc}[1]{\par\hspace*{-2em}\luainline{#1}\leftskip=2em\par}{\par}
%\newenvironment{APIparameters}[0]{\begin{description}[font=\mdseries\ttfamily]}{\end{description}}
\newenvironment{APIparameters}[0]{\par\noindent{}Parameters:\scriptsize\par\noindent\begin{tabular}{llp{0.5\linewidth}} Parameter & Type &
Explanation\\\midrule}{\end{tabular}} \def\parameter #1#2#3;{\texttt{#1} & #2 & #3 \\}
\def\parameterdefault #1#2#3#4;{\texttt{#1} (default #2) & #3 & #4 \\}

\usepackage{siunitx}
\usepackage{csquotes}

\frenchspacing

