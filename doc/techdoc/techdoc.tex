\documentclass[parskip=half]{scrartcl}

\usepackage{tikz}
\usetikzlibrary{positioning, shapes, patterns}
\tikzset{
    state/.style = {draw, align=center, minimum width = 2.1cm, rectangle split, rectangle split parts = 3, thick, minimum height = 5cm},
    tip/.style = {thick, ->, >=stealth}
}
\usepackage{wrapfig}

\usepackage{fontspec}
\usepackage{unicode-math}
\usepackage{microtype}
\setmainfont{Libertinus Serif}
\setsansfont{Libertinus Sans}
\setmathfont{Libertinus Math}

% maths
\usepackage{IEEEtrantools}

% tables
\usepackage{booktabs}

\usepackage{listings}
\lstset{
    breaklines = true,
    %breakatwhitespace = true,
    language=[5.3]Lua,
    backgroundcolor = \color{blue!10!white},
    basicstyle = \small\ttfamily,
    keywordstyle = \color{blue},
    commentstyle = \color{red},
    stringstyle = \color{green!70!black},
    showstringspaces = false,
    tabsize = 4,
    gobble = 4,
    emph = {
        layout,
        parameters,
        pointarray,
        path,
        geometry.rectangle, 
        geometry.multiple, 
        geometry.path, 
        pcell.setup, 
        pcell.process_args, 
        pcell.check_args,
        pcell.add_parameters,
        pcell.get_parameters,
        pcell.push_overwrites,
        pcell.pop_overwrites,
        object.translate,
        object.merge_into_shallow,
        get_anchor,
        move_anchor,
        translate,
        merge_into_shallow,
        flipx,
        flipy,
        add_child,
        add_child_reference,
        add_child_link,
        generics,
        generics.metal,
        generics.via,
        generics.contact,
        generics.other,
        generics.mapped,
        util.xmirror,
        util.make_insert_xy
    },
    emphstyle = \color{blue!60!green}\bfseries,
    belowskip=-0.2\baselineskip
}
% -------------------------------------------------------------------------------------------------------------
% taken from https://tex.stackexchange.com/questions/48903/how-to-extend-the-lstinputlisting-command
\errorcontextlines=\maxdimen

\newlength{\rawgobble}
\newlength{\gobble}
% Remove a single space
\settowidth{\rawgobble}{\small\ttfamily\ }
\setlength{\rawgobble}{-\rawgobble}

\makeatletter
\lst@Key{widthgobble}{0}{%
    % Reindent a bit by multiplying with 0.9, then multiply by tabsize and number of indentation levels
    %\setlength{\gobble}{0.9\rawgobble}%
    \setlength{\gobble}{#1\rawgobble}%
    %\setlength{\gobble}{\sepstartwo\gobble}%
    \def\lst@xleftmargin{\gobble}%
    \def\lst@framexleftmargin{\gobble}%
}
\makeatother
% -------------------------------------------------------------------------------------------------------------
\newcommand{\shellinline}[1]{\lstinline!#1!}
%\lstnewenvironment{shellcode}{\lstset{language=bash, keywordstyle = \relax, commentstyle = \relax, stringstyle = \relax, identifierstyle = \relax, breakautoindent = false}}{}
\lstnewenvironment{shellcode}{\lstset{keywordstyle = \relax, commentstyle = \relax, stringstyle = \relax, identifierstyle = \relax, breakautoindent = false}}{}
\newcommand{\luainline}[1]{\lstinline!#1!}
\newcommand{\lualisting}[2][]{\lstinputlisting[#1]{#2}}
%\newmintedfile[lualisting]{lua}{autogobble, fontsize = \small}
% API documentation commands
\newcommand{\param}[1]{\luainline{#1}}
\newenvironment{apifunc}[1]{\par\hspace*{-2em}\luainline{#1}\leftskip=2em\par}{\par}

\usepackage{siunitx}
\usepackage{csquotes}

\frenchspacing



\title{OpenPCells}
\subtitle{Technical Documentation and Implementation Notes}
\author{Patrick Kurth}

\begin{document}
\maketitle
\section{Technology Mapping}
Mapping from generic cell descriptions to technology-specific data has to perform several steps:
\begin{itemize}
    \item resolve relative metal numbering
    \item split up via stacks
    \item translate via rectangles to via arrays
    \item map all remaining layers \footnote{The via translation already generates technology-specific layers.}
\end{itemize}

Figure \ref{fig:techtranslation} shows the technology translation from generic to specific cells. This example technology has 7 metal layers, therefor "M-2" points
to "M6".
\begin{figure}[htb]
    \centering
    \begin{tikzpicture}
        [
            node distance = 2cm
        ]
        \node[state]                        (initial)  {Shape \nodepart{two} Polygon: \\"M-2" \nodepart{three} Via: \\"viaM-2M-4"};
        \node[state, right = of initial]    (metal)    {Shape \nodepart{two} Polygon: \\"M6"  \nodepart{three} Via: \\"viaM4M6"};
        \node[state, right = of metal]      (via)      {Shape \nodepart{two} Polygon: \\"M6" \nodepart{three} Via: \\"viaM4M5"\\"viaM5M6"};
        \node[state, right = of via]        (layer)    {Shape \nodepart{two} Polygon: \\"met6" \nodepart{three} Via: \\"V4"\\"V5"};
        % arrows
        \draw[tip]  (initial) -- node[above, align = center] {metal\\ translation} (metal);
        \draw[tip]  (metal)   -- node[above, align = center] {via\\ splitting}     (via);
        \draw[tip]  (via)     -- node[above, align = center] {layer\\ mapping} node[below, align = center] {via\\ arrayzation} (layer);
    \end{tikzpicture}
    \caption{Technology translation}
    \label{fig:techtranslation}
\end{figure}

\subsection{Metal Numbering}
For some cells like inductors it is customary to specify things like \emph{last metal} or a metal relative to another. This has to be resolved for further
processing, which is done in this step. Currently, only negative numbers (such as "M-1") are being processed into something like "M8" (depending on the total number
of metals in the technology).
\subsection{Via Splitting}
It is allowed the create via stacks, that is vias with non-adjacent metals. These have to split up into several shapes before via arrayzation.
\subsection{Via Arrayzation}
Via geometries can't be inside generic PCells, since these vary from technology to technology. For this reason, only rectangular areas where vias are to be placed in
a cell are specified. The technology translation then must create the actual via shapes, as shown in figure \ref{fig:viatranslation}.
\begin{wrapfigure}{r}{6cm}
    \centering
    \begin{tikzpicture}[scale = 2]
        \def\viaxsize{0.2}
        \def\viaysize{0.2}
        \draw[thick, pattern = north west lines] (-0.4, -1) rectangle (0.4, 1);
        \draw[densely dotted] (1.6, -1) rectangle (2.4, 1);
        \foreach \x in {-0.15, 0.15}
        {
            \foreach \y in {-0.8, -0.4, 0.0, 0.4, 0.8}
            {
                \draw[thick, pattern = crosshatch dots] 
                    ({2 + \x - 0.5 * \viaxsize}, {\y - 0.5 * \viaysize}) rectangle ({2 + \x + 0.5 * \viaxsize}, {\y + 0.5 * \viaysize});
            }
        }
        \draw[tip] (0.5, 0) -- node[above, align = center] { Generate \\ Via Array } (1.5, 0);
        % legend
        \draw[thick, pattern = north west lines] (-0.4, -1.4) rectangle (-0.2, -1.2);
        \node[anchor = west] at (-0.1, -1.3) {Generic Layer};
        \draw[thick, pattern = crosshatch dots] (-0.4, -1.7) rectangle (-0.2, -1.5);
        \node[anchor = west] at (-0.1, -1.6) {Specific Layer};
    \end{tikzpicture}
    \caption{Example of via arrayzation}
    \label{fig:viatranslation}
\end{wrapfigure}
\subsection{Layer Mapping}

\end{document}

% vim: ft=tex
